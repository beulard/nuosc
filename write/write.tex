\documentclass[10pt, a4paper]{article}

% for braket notation
\usepackage{physics}

\newcommand{\me}{\mathrm{e}}
\newcommand{\mi}{\mathrm{i}}

\renewcommand\refname{Bibliography}


\title{Project}
\author{Matthias Dubouchet}
\date{}


\begin{document}
\maketitle

\subsection{Neutrino oscillation formalism} 
Let us consider a general neutrino state as a superposition of $n$ orthogonal
eigenstates \cite{zuber}. We have two obvious eigenbases: the flavour (weak) eigenstates
$\ket{\nu_{\alpha}}$ and the mass eigenstates $\ket{\nu_i}$. Flavour
eigenstates interact with matter through the weak interaction,
and hence are the states we observe in nature.
Eigenstates are orthogonal within each basis and the two bases are related by a
unitary rotation $U$: 
\begin{align*}
\bra{\nu_\alpha}\ket{\nu_\beta} = \delta_{\alpha \beta}, \quad
\bra{\nu_i}\ket{\nu_j} = \delta_{ij}  \quad
\ket{\nu_\alpha} = \sum_i U_{\alpha i} \ket{\nu_i}\\
\ket{\nu_i} = \sum_i (U^\dagger)_{i \alpha} \ket{\nu_\alpha} = \sum_i
	U^*_{\alpha i} \ket{\nu_\alpha}, U^\dagger U = 1.
\end{align*}
Mass eigenstates are stationary states and evolve in time as $\ket{\nu_i(x,
t)} = \me^{-\mi E_i t} \ket{\nu_i(x, 0)}$, where $\ket{\nu_i(x, 0)} = \me^{\mi p
x} \ket{\nu_i}$ for a plane wave neutrino produced at $x=0$.
Hence a neutrino produced as a flavour eigenstate $\alpha$ would show the
following time dependence:
\begin{align*}
\ket{\nu(x, t)} &= \sum_i U_{\alpha i} \ket{\nu_i(x, t)} \\
		&= \sum_i U_{\alpha i} \me^{-\mi E_i t} \me^{\mi p x} \ket{\nu_i}\\
		&= \sum_{i, \beta} U_{\alpha i} U^*_{\beta i} \me^{\mi p x}
				\me^{-\mi E_i t} \ket{\nu_\beta},
\end{align*}
hence the transition amplitude to a flavour $\beta$ is 
\begin{align}
A(\alpha \rightarrow \beta) = \bra{\nu_\beta}\ket{\nu(x, t)} \nonumber
			= \sum_i U^*_{\beta i} U_{\alpha i} \me^{\mi p x}
			\me^{-\mi E_i t}.
\end{align}



Thanks to their small masses (regardless of which eigenstate),
neutrinos are highly relativistic particles so we can perform a binomial
approximation of their energy:
$$ E_i = \sqrt{m_i^2 + p_i^2} \simeq p_i + \frac{m_i^2}{2 p_i} \simeq E +
\frac{m_i^2}{2E},$$
where we have defined $E \approx p$ as the neutrino energy at the source. Using
this,
$$A(\alpha \rightarrow \beta) = \sum_i
U^*_{\beta i} U_{\alpha i} \exp(-\mi \frac{m_i^2 L}{2E}),$$
where $L = x \simeq c t$ is the distance from the source to the detector, using the
fact that neutrinos travel close to the speed of light.
The transition probability is then
\begin{align}
	\label{eq:nuprob}
	P(\alpha \rightarrow \beta) &= |A(\alpha \rightarrow \beta)|^2 \nonumber\\
	&= \sum_i \sum_j U_{\alpha i} U^*_{\beta i} U^*_{\alpha j} U_{\beta j}
	\exp(-\mi \frac{\Delta m_{ij}^2}{2} \frac{L}{E}),
\end{align}
with $\Delta m^2_{ij} = m^2_i - m^2_j$.
The analysis is similar when considering antineutrinos except that $U$ must be
replaced by $U^*$ everywhere and vice-versa.



\subsection{Two neutrino oscillations} 
The case for two neutrino eigenstates is the simplest to consider. The
unitary nature of the mixing matrix $U$ and the arbitrariness of the complex phases
of quantum states reduces the number of observable mixing parameters to one.
The relationship between flavour and mass eigenstates can be written as 
\begin{equation}
	\begin{bmatrix} \nu_e \\ \nu_\mu \end{bmatrix} =
	\begin{bmatrix} \cos\theta & \sin\theta \\
								 -\sin\theta & \cos\theta \end{bmatrix}
	\begin{bmatrix} \nu_1 \\ \nu_2 \end{bmatrix},
\end{equation}
and the transition probability follows from equation \ref{eq:nuprob}:
	$$P(\nu_e \rightarrow \nu_\mu) = P(\nu_mu \rightarrow \nu_e) = \sin^2 2\theta
	\sin^2 \frac{\Delta m^2}{4} \frac{L}{E}.$$

\subsection{Three neutrino oscillations}
When we consider three eigenstates, the number of parameters jumps to four:
three CP-preserving mixing angles and a CP-violating phase $\delta_{CP}$. The
mixing matrix was introduced by Maki, Nakagawa and Sakata in 1962~\cite{MNS}.
Analogously to the two neutrino case, it can be presented as a rotation matrix,
with the mixing angles $\theta_{ij}$ acting as Euler angles and the
CP-violating phase 
% why does the phase appear where it does specifically? read  Chau, Keung paper
$$
U_{\mathrm{MNS}} = 
\begin{bmatrix} 1 & 0 & 0 \\ 0 & c_{23} & s_{23} \\ 0 & -s_{23} & c_{23} \end{bmatrix}
\begin{bmatrix} c_{13} & 0 & s_{13} \me^{-\mi \delta_{CP}} \\ 0 & 1 & 0 \\
								-s_{13} \me^{\mi \delta_{CP}} & 0 & c_{13} \end{bmatrix} 
\begin{bmatrix} c_{12} & s_{12} & 0 \\ -s_{12} & c_{12} & 0 \\ 0 & 0 & 1 \end{bmatrix}
,$$
where $c_{ij} = \cos(\theta_{ij})$ and $s_{ij} = \sin(\theta_{ij})$.

\subsection{CP violating phase}

\subsection{Sensitivities}


\subsubsection{Delta chi squared} 
Let's work under the assumption that the true mass hierarchy (the one chosen by
nature) is the \emph{normal} one. When we oscillate our initial spectrum over
the baseline and obtain a prediction for the far detector spectrum, we are,
under this assumption, modeling the true oscillations as they will happen in
the real experiment. Since the CP violating phase $\delta_{CP}$ is still
experimentally unknown, it is useful to perform the neutrino oscillations while
changing $\delta_{CP}$ over its allowed range $[-\pi,\pi)$. For each of these
spectra, we are thus making two assumptions: one is the true mass hierarchy,
and the other is the true CP violating phase.

Now we ask: if our assumptions are wrong and either the true mass hierarchy is
inverted, or the phase is not what we think, or both: how accurately can we
rule out our initial assumptions? To answer this, we repeat the oscillation
with inverted hierarchy parameters, and again, we explore every value of
$\delta_{CP}$. A good way to define the agreement between two sets of data (or
in this case, predictions) is the $\chi^2$ statistic. We calculate a $\chi^2$
between our \emph{assumed true} spectrum and each of the \emph{inverted
hierarchy} spectra. The smaller the $\chi^2$, the better the fit, hence the
minimum $\chi^2$, for a given assumed $\delta_{CP}$, represents the best
possible agreement with the opposite hierarchy. This particular statistic is
called the mean $\Delta\chi^2$ ($\overline{\Delta\chi^2}$), and since it tells
us how ``badly'' our experiment might differentiate between different values of
a parameter,  it is a good measure of the \emph{sensitivity} of the experiment
to this underlying parameter.  We call the quantity
$\sqrt{\overline{\Delta\chi^2}}$ the sensitivity.


\begin{thebibliography}{20} 
	\bibitem{zuber} Kai Zuber, \textit{Neutrino
		Oscillations}, 2004
	\bibitem{MNS} Maki, Nakagawa, Sakata, \textit{Remarks on the Unified Model of
		Elementary Particles}, 1962
	
\end{thebibliography}

\end{document}
