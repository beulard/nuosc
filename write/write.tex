\documentclass[10pt, a4paper]{article}

% for braket notation
\usepackage{physics}

\newcommand{\me}{\mathrm{e}}
\newcommand{\mi}{\mathrm{i}}

\renewcommand\refname{Bibliography}


\title{Project}
\author{Matthias Dubouchet}
\date{}


\begin{document}
\maketitle


\subsection{Neutrinos and neutrino oscillations}
The neutrino particle first appeared in 1934 in Fermi's theory of the $\beta$-decay
process\cite{fermi}, where a neutron would decay into a proton and emit an electron and a
light, neutral particle. This extra particle was required as a way to explain the
continuous energy spectrum of the electron. A number of subsequent experiments
were designed to map out the properties of the neutrino and its
interactions\cite{zuber}. As of today, the neutrino is still not fully
understood, and neutrino physics has become one of the leading branches of
experimental particle physics along with collider physics. 

The existence of distinct flavours of neutrinos was first investigated by
Bruno Pontecorvo in 1959\cite{pontecorvo}. He found that different neutrino flavours
interacted with different leptons, therefore establishing the three neutrinos
known today as $\nu_e$, $\nu_\mu$ and $\nu_\tau$, and hinting at the
possibility of flavour oscillations, which he later theorized.
As will be discussed later, the fact that a neutrino can change its flavour by
propagating through empty space is direct evidence that neutrinos are massive
particles, unlike what is initially described by the Standard Model.
The 2015 Nobel Prize was awarded to the SuperKamiokande and SNO collaborations
for the discovery of neutrino oscillations in 1999 and 2001, respectively.
Although neutrino oscillations have been observed, some of the physical parameters that
describe this process have remained difficult to probe because of the lack of
accuracy in the measurements of past experiments. 
The Deep Underground Neutrino Experiment (DUNE) in the United States and the
upgrade to SuperKamiokande, HyperKamiokande (HyperK) in Japan, are examples of
future experiments that aim to determine these parameters and start the era of
precision measurements in neutrino physics.

In the Standard Model, there are twelve elementary fermions, six quarks and six
leptons (plus their respective antiparticles). The leptons consist of three
charged leptons $(e, \mu, \tau)$ and three corresponding charge-less neutrinos
$(\nu_e, \nu_\mu, \nu_\tau)$. The electron neutrino $\nu_e$, for example, is
\emph{defined} as the neutrino which can interact weakly with an electron and a
$W$ boson~(Fig.~\ref{fig:weak_vertex}).

\begin{figure}
	\label{fig:weak_vertex}
\end{figure}

Neutrinos, being leptons and carrying zero charge, interact only via the weak
interaction, which makes them impossible to observe directly. Their existence
and their properties can only be inferred from their interactions with other
particles.
Experimentally, only left-handed neutrinos and right-handed
antineutrinos are observed. In the Standard Model, this curious property would
prevent the neutrinos from acquiring a mass through interactions with the
Higgs field in the same way that charged leptons do. Hence the Standard Model
on its own cannot accurately describe neutrinos, and extensions such as the
seesaw mechanism or extra dimensions are needed. This discussion lies beyond
the scope of this paper since neutrino oscillations can be described by fairly
straightforward matrix and quantum mechanics, and quantum field theory will not
be needed. % mal dit

\section{}
We will now introduce the mathematical tools that we use to describe neutrino
oscillations. Neutrinos are able to oscillate flavour because the quantum states that
are produced in weak interactions, called the \emph{weak eigenstates}, are
different from the states that propagate in spacetime under the free
Hamiltonian, called the \emph{mass eigenstates}. As we will see, these two
eigenbases are simply related by a \emph{mixing matrix}, analogous (or, in the
case of two neutrinos, identical) to a rotation of the axes in a Cartesian
coordinate system.
The derivations in this section are heavily inspired by chapter 8 of
Zuber's textbook on neutrino physics\cite{zuber}.

\subsection{Neutrino oscillation formalism} 
Let us consider a general neutrino state as a superposition of $n$ orthogonal
eigenstates. We mentioned the two eigenbases of interest: the weak (flavour)
eigenstates $\ket{\nu_{\alpha}}$, $\alpha=e, \mu, \tau$, and the mass
eigenstates $\ket{\nu_i}$, $i=1, 2, 3$. 
This is stupid, we are specifying 3 eigenstates when we said we were
considering the general case.
Flavour eigenstates interact with
matter through the weak interaction, and hence are the states we observe in
nature.  Eigenstates are orthogonal within each basis and the two bases are
related by a unitary transformation conventionally represented by the
Pontecorvo-Maki-Nakagawa-Sakata (PMNS) mixing (((Citation needed)))
matrix $U$: 
\begin{align*}
\bra{\nu_\alpha}\ket{\nu_\beta} = \delta_{\alpha \beta}, \quad
	\bra{\nu_i}\ket{\nu_j} &= \delta_{ij}  \quad
\ket{\nu_\alpha} = \sum_i U_{\alpha i} \ket{\nu_i}\\
	\ket{\nu_i} = \sum_\gamma (U^\dagger)_{i \gamma}~\quad &\quad \ket{\nu_\gamma} = \sum_\gamma
	U^*_{\gamma i} \ket{\nu_\gamma} \\U^\dagger U &= 1,
\end{align*}
where $\alpha, \beta, \gamma = e, \mu, \tau$ and $i, j = 1, 2, 3$. ((Pas clair,
on dirait que alpha, beta, gamma are respectively e, mu and tau.))\\
Since they are solutions of the free Hamiltonian in a vacuum, the mass eigenstates are stationary states and evolve in time as $\ket{\nu_i(x,
t)} = \me^{-\mi E_i t} \ket{\nu_i(x, 0)}$, where $\ket{\nu_i(x, 0)} = \me^{\mi p
x} \ket{\nu_i}$ for a plane wave neutrino produced at $x=0$.
Hence a neutrino produced as a flavour eigenstate $\alpha$ would show the
following time dependence:
\begin{align*}
\ket{\nu(x, t)} &= \sum_i U_{\alpha i} \ket{\nu_i(x, t)} \\
		&= \sum_i U_{\alpha i} \me^{-\mi E_i t} \me^{\mi p x} \ket{\nu_i}\\
		&= \sum_{i, \beta} U_{\alpha i} U^*_{\beta i} \me^{\mi p x}
				\me^{-\mi E_i t} \ket{\nu_\beta},
\end{align*}
hence the transition amplitude to a flavour $\beta$ is 
\begin{align}
A(\alpha \rightarrow \beta) = \bra{\nu_\beta}\ket{\nu(x, t)} \nonumber
			= \sum_i U^*_{\beta i} U_{\alpha i} \me^{\mi p x}
			\me^{-\mi E_i t}.
\end{align}



Thanks to their small masses (regardless of which eigenstate),
neutrinos are highly relativistic particles so we can perform a binomial
approximation of their energy:
$$ E_i = \sqrt{m_i^2 + p_i^2} \simeq p_i + \frac{m_i^2}{2 p_i} \simeq E +
\frac{m_i^2}{2E},$$
where we have defined $E \approx p$ as the neutrino energy at the source. Using
this,
$$A(\alpha \rightarrow \beta) = \sum_i
U^*_{\beta i} U_{\alpha i} \exp(-\mi \frac{m_i^2 L}{2E}),$$
where $L = x \simeq c t$ is the distance from the source to the detector, using the
fact that neutrinos travel close to the speed of light.
The transition probability is then
\begin{align}
	\label{eq:nuprob}
	P(\alpha \rightarrow \beta) &= |A(\alpha \rightarrow \beta)|^2 \nonumber\\
	&= \sum_i \sum_j U_{\alpha i} U^*_{\beta i} U^*_{\alpha j} U_{\beta j}
	\exp(-\mi \frac{\Delta m_{ij}^2}{2} \frac{L}{E}),
\end{align}
with $\Delta m^2_{ij} = m^2_i - m^2_j$.
The analysis is similar when considering antineutrinos except that $U$ must be
replaced by $U^*$ everywhere and vice-versa.



\subsection{Two neutrino oscillations} 
The case for two neutrino eigenstates is the simplest to consider. The
unitary nature of the mixing matrix $U$ and the arbitrariness of the complex phases
of quantum states reduces the number of observable mixing parameters to one.
The relationship between flavour and mass eigenstates can be written as 
\begin{equation}
	\begin{bmatrix} \nu_e \\ \nu_\mu \end{bmatrix} =
	\begin{bmatrix} \cos\theta & \sin\theta \\
								 -\sin\theta & \cos\theta \end{bmatrix}
	\begin{bmatrix} \nu_1 \\ \nu_2 \end{bmatrix},
\end{equation}
and the transition probability follows from equation \ref{eq:nuprob}:
	$$P(\nu_e \rightarrow \nu_\mu) = P(\nu_mu \rightarrow \nu_e) = \sin^2 2\theta
	\sin^2 \frac{\Delta m^2}{4} \frac{L}{E}.$$

\subsection{Three neutrino oscillations}
When we consider three eigenstates, the number of parameters jumps to four:
three CP-preserving mixing angles and a CP-violating phase $\delta_{CP}$. The
mixing matrix was introduced by Maki, Nakagawa and Sakata in 1962\cite{MNS}.
Analogously to the two neutrino case, it can be presented as a rotation matrix,
with the mixing angles $\theta_{ij}$ acting as Euler angles and the
CP-violating phase 
% why does the phase appear where it does specifically? read  Chau, Keung paper
$$
U_{\mathrm{MNS}} = 
\begin{bmatrix} 1 & 0 & 0 \\ 0 & c_{23} & s_{23} \\ 0 & -s_{23} & c_{23} \end{bmatrix}
\begin{bmatrix} c_{13} & 0 & s_{13} \me^{-\mi \delta_{CP}} \\ 0 & 1 & 0 \\
								-s_{13} \me^{\mi \delta_{CP}} & 0 & c_{13} \end{bmatrix} 
\begin{bmatrix} c_{12} & s_{12} & 0 \\ -s_{12} & c_{12} & 0 \\ 0 & 0 & 1 \end{bmatrix}
,$$
where $c_{ij} = \cos(\theta_{ij})$ and $s_{ij} = \sin(\theta_{ij})$.

\subsection{Neutrino oscillations in matter}
So far we have only discussed neutrino oscillations in vacuum, where they do
not interact. In matter, where atoms are present, they can interact with
electrons and nucleons (?) via the weak interaction [citation needed], which affects the
oscillations. A neutrino can scatter either by exchanging a $W^\pm$ boson ---
charged current (CC) interaction --- or by exchanging a $Z$ boson --- neutral
current (NC) interaction. The neutral current interaction is experienced
equally by all three flavours, hence it does not have an effect on the relative
propagation phases, and causes no change in the oscillation probability. The
charged current is, as we have seen from the interaction vertex
(fig.~\ref{fig:weak_vertex}), proper (?) to the neutrino flavour. Since only
electrons are present in matter, only electron neutrinos are subject to CC
interactions, and oscillation probabilities

((CAN SHOW DEMONSTRATION FOR 2 FLAVOURS, SEE ZUBER p.216 OR RICCIARDI'S
NOTES))


Experiments in which neutrinos spend a significant amount of time propagating through
matter --- e.g. neutrinos from the sun's core, or neutrinos propagating underground in the
Earth's mantle --- require us to take matter effects into account in order to produce a
realistic model. The DUNE experiment will be performed entirely underground
with a baseline of 1300km, enough to have a significant effect on the
oscillation probability from muon neutrinos to electron neutrinos. A comparison
between models with and without matter effects for that baseline is shown on
figure~\ref{fig:matter_effects}.
((TODO PLOT MATTER.C))

\subsection{CP violating phase}

\subsection{Sensitivities}
Neutrino experiments are characterised by the extremely small scattering cross
section of weak processes. Neutrinos are almost massless, have no charge, and
being leptons, they are not affected by strong interactions. These unique
properties make their detection particularly difficult and the associated
experiments particularly costly and time-inefficient. The best way for
experimenters to compensate for a small cross section is to increase the scale
of their detector. 

The Sudbury Neutrino Observatory (SNO) experiment, which confirmed
the existence of matter effects on neutrino oscillations in 2001, consisted of 1000
tons of heavy water enclosed in a sphere of diameter 12 meters\cite{thomson}.
The DUNE experiment, expected to start collecting data in 2024, will use 4850L
of liquid argon for a total fiducial mass\footnote{The outermost parts of the
detector are more sensitive to background events. To account for this, all
events outside the more reliable central region are discarded. The part of the
total detector mass that remains is called fiducial mass.} of 40 kilotons\cite{cdr}.
For projects of this scale, it is crucial to try to maximize the time and cost
efficiency of the experiment by choosing the design that will provide the best
chance of scientific discovery.

In general, it is not obvious which design choices (baseline, for example) will lead to better
sensitivity. By sensitivity, we mean the ability of the experiment to make a
relevant discovery. For example, if DUNE was to estimate $\delta_{CP}$ and
determine that it excludes $\delta_{CP}=0$ and $\delta_{CP}=\pi$ with a
$3\sigma$ confidence level, one could say it has discovered CP violation in the
lepton sector.

As a consequence, experimentalists have devised statistical methods to allow
quantifying of sensitivity in order to compare different designs,
and create the most efficient experiments.
Additionally, these methods enable us to compare experiments that have
different characteristics --- baseline, detector technology, neutrino source
--- and to understand the complementarities that could result from combining
their results.


One of the most sought after discoveries in neutrino physics is that of the neutrino
mass hierarchy, that is, the sign of the mass difference $\Delta m^2_{31} =
m^2_3 - m^2_1$. Naturally, we can define a statistic that describes how
sensitive a given experiment is to the mass hierarchy. That is, this
statistic tells us how good the experiment will be at discriminating
between the two possible hypotheses, NH or IH.

((Give a heuristic introduction to the $\Delta \chi^2$ statistic and how it is
used to estimate sensitivity))
((Very good inspiration in Introduction from Ciuffoli 1305.5150))

\subsubsection{Derivation of $\Delta \chi^2$ for the mass hierarchy}
The mass hierarchy can only turn out to be normal (NH), or inverted (IH). As
such, it is considered as a discrete hypothesis (in contrast with a continuous
variable such as $\delta_{CP}$ or $\theta_{23}$) in our statistical analysis.
The method we use\cite{ciuffoli, qian} is presented in a fairly
statistics-heavy manner but we only use it in its simplest occurence,
namely a model with no nuisance parameters. However, we later introduce
systematic uncertainties by hand by modeling the reconstruction of neutrino
energies at the detector.
In a neutrino experiment such as DUNE, neutrino events (e.g.~$\nu_e$
appearance) will be recorded as a
function of their reconstructed energy. We denote by $y_i$ the event count,
where $i = 1, 2, ..., N$ is the index of the energy bin. The event counts are
distributed according to a Gaussian with variance $\sigma_i^2$. Our theoretical models
will yield predicted event counts that will depend on the true mass hierarchy.
We denote by $y^N_i$ the predicted count under normal hierarchy and by $y^I_i$
the predicted count under inverted hierarchy.
Let us assume that the true hierarchy is normally ordered. Then we can write
our experimental event counts as $y_i = y^N_i + \sigma_i g_i$, where $g_i$ is a
random variable from a Gaussian distribution.
The $\Delta \chi^2$ statistic can then be written as
\begin{align*}
	\Delta \chi^2 &= \chi_I^2 - \chi_N^2\\
	&= \sum_i \frac{(y_i - y^I_i)^2}{\sigma_i^2} - \sum_i \frac{(y_i -
	y^N_i)^2}{\sigma_i^2}\\
	&= \sum_i \frac{(y^N_i + \sigma_i g_i - y^I_i)^2 - (y^N_i + \sigma_i g_i -
	y^N_i)^2}{\sigma_i^2}\\
	&= \sum_i \frac{(y^N_i - y^I_i)^2}{\sigma_i^2} + \sum_i \frac{2(y^N_i -
	y^I_i)}{\sigma_i} g_i
\end{align*}
In this form, we can see that $\Delta \chi^2$ is a Gaussian-distributed random
variable where the mean is
$$\overline{\Delta\chi^2} = \sum_i \frac{(y^N_i - y^I_i)^2}{\sigma_i^2},$$
and the standard deviation is
$$\sigma_{\Delta\chi^2} = \sqrt{\sum_i \frac{4(y^N_i - y^I_i)^2}{\sigma_i^2}}\\
= 2\sqrt{\overline{\Delta\chi^2}}.$$

In these expressions, the experimental counts $y_i$ do not appear, which means
that we can evaluate the $\overline{\Delta\chi^2}$ (mean delta chi-squared)
statistic from our theoretical models only.

((TODO this is not true, fix))
It is fairly evident\cite{qian} that when the
event count $y^N_i$ is large (we are still under our assumption that the true hierarchy
is normal), we can substitute the statistical uncertainty $\sigma_i$ by
$\sqrt{y^N_i}$, such that our expression becomes
$$\overline{\Delta\chi^2} = \sum_i \frac{(y^N_i - y^I_i)^2}{y^N_i}.$$
If we work under the opposite assumption, where we have a true inverted
hierarchy, the denominator is simply replaced by $y^I_i$. 

This statistic is used throughout the neutrino physics literature\cite{cdr,
martin-albo}((MORE CITATIONS NEEDED)) to express the sensitivity of future experiments.
((Give details)).



\subsubsection{Sensitivity to $\delta_{CP}$}
In addition to the sensitivity to the mass hierarchy, we estimate the
sensitivity to CP discovery by simulating data for all values of $\delta_{CP}$
and comparing to CP-conserving values $\delta_{CP} = 0, \pi$.



\subsubsection{Delta chi squared} 
((TODO read this section and retain good bits))
Let's work under the assumption that the true mass hierarchy (the one chosen by
nature) is the \emph{normal} one. When we oscillate our initial spectrum over
the baseline and obtain a prediction for the far detector spectrum, we are,
under this assumption, modeling the true oscillations as they will happen in
the real experiment. Since the CP violating phase $\delta_{CP}$ is still
experimentally unknown, it is useful to perform the neutrino oscillations while
changing $\delta_{CP}$ over its allowed range $[-\pi,\pi)$. For each of these
spectra, we are thus making two assumptions: one is the true mass hierarchy,
and the other is the true CP violating phase.

Now we ask: if our assumptions are wrong and either the true mass hierarchy is
inverted, or the phase is not what we think, or both: how accurately can we
rule out our initial assumptions? To answer this, we repeat the oscillation
with inverted hierarchy parameters, and again, we explore every value of
$\delta_{CP}$. A good way to define the agreement between two sets of data (or
in this case, predictions) is the $\chi^2$ statistic. We calculate a $\chi^2$
between our \emph{assumed true} spectrum and each of the \emph{inverted
hierarchy} spectra. The smaller the $\chi^2$, the better the fit, hence the
minimum $\chi^2$, for a given assumed $\delta_{CP}$, represents the best
possible agreement with the opposite hierarchy. This particular statistic is
called the $\overline{\Delta\chi^2}$, the mean $\Delta\chi^2$, and since it tells
us how ``badly'' our experiment might differentiate between different values of
a parameter,  it is a good measure of the \emph{sensitivity} of the experiment
to this underlying parameter.  We call the quantity
$\sqrt{\overline{\Delta\chi^2}}$ the sensitivity.

% TODO
% Derivation of delta chi squared sensitivity, quote paper 'Sensitivity to the
% Neutrino Mass Hierarchy.pdf'.


\begin{thebibliography}{20} 
	\bibitem{zuber} Zuber K. Neutrino
		Oscillations. \textit{IOP Publishing}. 2004
	\bibitem{thomson} Thomson M. Modern Particle Physics. \textit{Cambridge
		University Press}. 2013
	\bibitem{MNS} Maki Z, Nakagawa M, Sakata S. Remarks on the Unified Model of
		Elementary Particles. \textit{Progress of Theoretical Physics}. 1962
	\bibitem{cdr} DUNE Collaboration. Long-Baseline Neutrino Facility (LBNF) and
		Deep Underground Neutrino Experiment (DUNE) Conceptual Design Report Volume
		2: The Physics Program for DUNE at LBNF. \texttt{arXiv:1512.06148}. 2015
	\bibitem{fermi} Fermi E. Versuch einer Theorie der $\beta$-Strahlen. I.
		\textit{Z. Physik}. 1934
	\bibitem{pontecorvo} Pontecorvo B. \textit{Sov. J. Phys.} 1960
	\bibitem{ciuffoli} Ciuffoli E, Evslin J, Zhang X. Sensitivity to the
		Neutrino Mass Hierarchy. \texttt{arXiv:1305.5150v4}. 2013
	\bibitem{qian} Qian X, Tan A, Wang W, Ling JJ, McKeown RD, Zhang C.
		Statistical Evaluation of Experimental Determinations of Neutrino Mass
		Hierarchy. \texttt{arXiv:1210.3651}. 2012
	\bibitem{martin-albo} Martin-Albo J. Sensitivity of DUNE to long-baseline
		neutrino oscillation physics. \texttt{arXiv:1710.08964}. 2017
	
\end{thebibliography} \end{document}
