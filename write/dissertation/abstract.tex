\begin{center}

\vspace*{2cm} % adjust as you wish
        
{\large\textbf{Abstract}}
        
\vspace{1cm} % adjust as you wish
\end{center}

Neutrino oscillation physics has become a major branch of experimental particle
physics over the last twenty years, after the observation of flavour
oscillations in 1999 and 2001 by the Super-Kamiokande and Sudbury Neutrino
Observatory experiments, respectively. Currently, three of the six oscillation
parameters are known within a few percent, while the other three are expected
to be determined by the current or next generation of experiments.
A noteworthy next generation neutrino experiment is the Deep Underground
Neutrino Experiment (DUNE), in South Dakota. 

In the present dissertation, we review
the phenomenology of neutrino oscillations, produce a C++ implementation
thereof, and evaluate the sensitivity of the DUNE and Hyper-Kamiokande
(Hyper-K) projects to two of the oscillation parameters, namely the mass
hierarchy $\Delta m^2_{32}$ and the CP-violating phase.

We predict that DUNE, with an exposure of 150 kt$\cdot$MW$\cdot$year, will
reject one of the two mass hierarchies with a significance of at least
$5\sigma$ even for the least favourable combination of the true parameters.
Under either hierarchy, DUNE will reject CP conservation for about 10\% of
the possible true values of $\delta_{CP}$.

In addition, we predict that Hyper-K, with an exposure of 10 years, has a low
sensitivity to the mass hierarchy, with a maximum rejection significance of
about $3\sigma$. 
However we expect its sensitivity to CP violation to be very high, rejecting CP
conservation for 75\% of all possible values of $\delta_{CP}$, under either
hierarchy.



