\documentclass{beamer}
\usetheme{metropolis}
\usepackage{physics}
\usepackage{ragged2e}
\usepackage[absolute,overlay]{textpos}

\newcommand{\me}{\mathrm{e}}
\newcommand{\mi}{\mathrm{i}}

\title{Final year project presentation\\Neutrino
oscillations and \\experimental sensitivity}
\author{Matthias Dubouchet\\Dr. Lisa Falk}
\date{\today}

\begin{document}

\maketitle


\begin{frame}{Outline}

\end{frame}


\begin{frame}{Neutrinos}

\begin{columns}

	\column[T]{0.5\textwidth}
Properties
		\begin{itemize}
		\item Fermion
		\item Very light
		\item Interacts via weak force, gravity
		\item ...
		\end{itemize}
	\column[T]{0.5\textwidth}
Oscillations
		\begin{itemize}
			\item Theorised by Bruno Pontecorvo (1957)
				\item Confirmed by the SNO and SuperK experiments
				\item 2015 Nobel prize
		\end{itemize}
\end{columns}
\end{frame}


\begin{frame}{Neutrinos}

// dirac equation, chiral representation, left handedness, majorana

\end{frame}



\begin{frame}{Oscillation formalism}

	Three neutrino eigenstates/generations:\\
	\begin{itemize}
		\item Flavor eigenbasis $\ket{\nu_e}, \ket{\nu_\mu}, \ket{\nu_\tau}$\\
		\item Mass eigenbasis ~~$\ket{\nu_1}, \ket{\nu_2}, \ket{\nu_3}$
	\end{itemize}

	Related by:
	$$
	\begin{bmatrix} \nu_e \\ \nu_\mu \\ \nu_\tau \end{bmatrix} =
	\begin{bmatrix} U_{e 1} & U_{e 2} & U_{e 3} \\ U_{\mu 1} & U_{\mu 2} & U_{\mu 3}
	\\ U_{\tau 1} & U_{\tau 2} & U_{\tau 3} \end{bmatrix}
	\begin{bmatrix} \nu_1 \\ \nu_2 \\ \nu_3 \end{bmatrix}
	$$
	...
	// plot eigenbases as 3d bases rotated by rotation $U$
\end{frame}

\begin{frame}{Oscillation formalism}
	// mass eigenstate propagation
	// transition probability?


\end{frame}

\begin{frame}{Simple version: two neutrinos}
	\begin{textblock*}{0.6\textwidth}(0.6\textwidth,1.1cm)
	\includegraphics[width=0.6\textwidth,angle=-90]{twonu4.pdf}
	\end{textblock*}
	
	\begin{textblock*}{0.4\textwidth}(1cm, 2.1cm)
		$$
	\begin{bmatrix} \nu_e \\ \nu_\mu \end{bmatrix} =
	\begin{bmatrix} \cos\theta & \sin\theta \\
								 -\sin\theta & \cos\theta \end{bmatrix}
		\begin{bmatrix} \nu_1 \\ \nu_2 \end{bmatrix}
	$$
	\end{textblock*}

	\vspace{2.5cm}
	$$
		P(\nu_\mu \rightarrow \nu_\tau) = \sin^2(2 \theta) \sin^2\bigg(\frac{\Delta
		m^2 L}{4 E}\bigg)
	$$
\end{frame}

\begin{frame}{Three neutrino case}

	\begin{textblock*}{\textwidth}(1cm, 2cm)
	// More parameters: $\delta_{CP}$, more angles, more mass differences
	// introduce mass hierarchy, draw mass levels
	\end{textblock*}
	 
	\begin{textblock*}{\textwidth}(0.5cm,3cm)
		
		$$
		\begin{bmatrix} \nu_e \\ \nu_\mu \\ \nu_\tau \end{bmatrix} = 
		\begin{bmatrix} 1 & 0 & 0 \\ 0 & c_{23} & s_{23} \\ 0 & -s_{23} & c_{23} \end{bmatrix}
		\begin{bmatrix} c_{13} & 0 & s_{13} \me^{-\mi \delta_{CP}} \\ 0 & 1 & 0 \\
										-s_{13} \me^{\mi \delta_{CP}} & 0 & c_{13} \end{bmatrix} 
		\begin{bmatrix} c_{12} & s_{12} & 0 \\ -s_{12} & c_{12} & 0 \\ 0 & 0 & 1 \end{bmatrix}
		\begin{bmatrix} \nu_1 \\ \nu_2 \\ \nu_3 \end{bmatrix}
		$$
	\end{textblock*}
	\begin{textblock*}{0.5\textwidth}(0.1\textwidth,6.3cm)
		$$
			P(\nu_\alpha \rightarrow \nu_\beta) \propto \sum_{i j} \exp(-i
			\frac{\Delta m^2_{i j} L}{2 E}) \qquad \qquad \Delta m^2_{ij} = m^2_i - m^2_j
		$$
	\end{textblock*}
	
	\pause
	\begin{textblock*}{0.5\textwidth}(0.4\textwidth,1.1cm)
		\includegraphics<+->[width=0.95\textwidth,angle=-90]{threenu.pdf}
	\end{textblock*}

\end{frame}



\begin{frame}{The importance of parameters}

	// maybe plot side by side of two spectra with different $d cp$ / MH
	\begin{textblock*}{0.5\textwidth}(0.05\textwidth,3.1cm)
		\includegraphics[width=0.75\textwidth,angle=-90]{parameters.pdf}
	\end{textblock*}

\end{frame}


\begin{frame}{Matter effects}

	// forward scattering in matter
	// effective mass
	// impact on transition probability
	// comparison plot ?
	\begin{textblock*}{0.65\textwidth}(0.05\textwidth,5.1cm)
		\includegraphics[width=0.95\textwidth]{matter.pdf}
	\end{textblock*}

\end{frame}


\begin{frame}{Sensitivity}

	// can we know whether a future experiment will yield useful results?
	// introduce statistical method, $\overline{\Delta \chi^2}$

\end{frame}


\begin{frame}{Sensitivity}

	// preliminary plots are ok
	// interpretation of plots

\end{frame}


\begin{frame}{Results so far}

	\begin{textblock*}{\textwidth}(1cm, 1.3cm)
		Sensitivity to the mass hierarchy\vspace{-0.25cm}
	\includegraphics[width=0.3\textwidth,angle=-90]{sens.pdf}
	\end{textblock*}
	\begin{textblock*}{\textwidth}(1cm, 5.5cm)
		Sensitivity to the CP violating phase\vspace{-0.25cm}
	\includegraphics[width=0.3\textwidth,angle=-90]{sens_cp.pdf}
	\end{textblock*}

\end{frame}


\begin{frame}{What's next}

	// 

\end{frame}


\begin{frame}{Summary}

\end{frame}



\end{document}
